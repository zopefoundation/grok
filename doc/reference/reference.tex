% Complete documentation on the extended LaTeX markup used for Python
% documentation is available in ``Documenting Python'', which is part
% of the standard documentation for Python.  It may be found online
% at:
%
%     http://www.python.org/doc/current/doc/doc.html

\documentclass{manual}
\RequirePackage[latin9]{inputenc}
\usepackage{graphicx}

\title{grok reference}

% Please at least include a long-lived email address;
% the rest is at your discretion.
\authoraddress{
    The grok team\\
    Email: $<$grok-dev@zope.org$>$
}

\date{\today}   % update before release!
                % Use an explicit date so that reformatting
                % doesn't cause a new date to be used.  Setting
                % the date to \today can be used during draft
                % stages to make it easier to handle versions.

\release{unreleased}      % release version; this is used to define the
                          % \version macro

\makeindex          % tell \index to actually write the .idx file

\begin{document}

\maketitle

    \begin{quote}
    ``Grok means to understand so thoroughly that the observer becomes a part
    of the observed --- merge, blend, intermarry, lose identity in group
    experience. It means almost everything that we mean by religion,
    philosophy, and science --- it means as little to us (because we are from
    Earth) as color means to a blind man.'' -- Robert A. Heinlein, Stranger in
    a Strange Land
    \end{quote}

\begin{abstract}
This is the grok reference documentation. It is organized by the Python
artefacts that implement the concepts.

Grok makes Zope 3 concepts more accessible for application developers. This
reference is not intended as introductory material for those concepts. Please
refer to the original Zope 3 documentation and the grok tutorial for
introductory material.
\end{abstract}

\tableofcontents

\chapter{Core}

The \module{grok} module defines a few functions to interact with grok itself.


\section{\function{grok.grok} -- Grok a package or module}

    \begin{funcdesc}{grok}{dotted_name}

    Grokking a package or module activates the contained components (like
    models, views, adapters, templates, etc.) and registers them with Zope 3's
    component architecture.

    The \var{dotted_name} must specify either a Python module or package
    that is available from the current PYTHONPATH.

    Grokking a module:

    \begin{enumerate}

        \item Scan the module for known components: models, adapters,
              utilities, views, traversers, templates and subscribers.

        \item Check whether a directory with file system templates
              exists (\file{<modulename>_templates}).  If it exists,
              load the file system templates into the template
              registry for this module.

        \item Determine the module context. 

        \item Register all components with the Zope 3 component architecture.

        \item Initialize schemata for registered models

    \end{enumerate}

    Grokking a package:

    \begin{enumerate}
        \item Grok the package as a module.

        \item Check for a static resource directory (\file{static})
          and register it if it exists.

        \item Recursively grok all sub-modules and sub-packages.

    \end{enumerate}

    \end{funcdesc}



\chapter{Components}

The \module{grok} module defines a set of components that provide basic Zope 3
functionality in a convenient way.

\section{\class{grok.Adapter}}

  Implementation, configuration, and registration of Zope 3 adapters.

  \begin{classdesc*}{grok.Adapter}
    Base class to define an adapter. Adapters are automatically registered when
    a module is "grokked".

    \begin{memberdesc}{context}
      The adapted object.
    \end{memberdesc}

  \begin{bf}Directives:\end{bf}

  \begin{itemize}
    \item[\function{grok.context(context_obj_or_interface)}] Maybe required.
    Identifies the type of objects or interface for the adaptation.

    If Grok can determine a context for adaptation from the module, this
    directive can be omitted. If the automatically determined context is not
    correct, or if no context can be derived from the module the directive is
    required.

    \item[\function{grok.implements(*interfaces)}] Required. Identifies the
    interface(s) the adapter implements.

    \item[\function{grok.name(name)}] Optional. Identifies the name used for
    the adapter registration. If ommitted, no name will be used.

    When a name is used for the adapter registration, the adapter can only be
    retrieved by explicitely using its name.

    \item[\function{grok.provides(name)}] Maybe required. If the adapter
    implements more than one interface, \function{grok.provides} is required to
    disambiguate for what interface the adapter will be registered.
  \end{itemize}
  \end{classdesc*}

  \begin{bf}Example:\end{bf}

  \begin{verbatim}
import grok
from zope import interface

class Cave(grok.Model):
    pass

class IHome(interface.Interface):
    pass

class Home(grok.Adapter):
    grok.implements(IHome)

home = IHome(cave)
  \end{verbatim}

  \begin{bf}Example 2:\end{bf}

  \begin{verbatim}
import grok
from zope import interface

class Cave(grok.Model):
    pass

class IHome(interface.Interface):
    pass

class Home(grok.Adapter):
    grok.implements(IHome)
    grok.name('home')

from zope.component import getAdapter
home = getAdapter(cave, IHome, name='home')
  \end{verbatim}

\section{\class{grok.AddForm}}

\section{\class{grok.Annotation}}

\section{\class{grok.Application}}

\section{grok.ClassGrokker}

\section{\class{grok.Container}}

  \begin{classdesc*}{grok.Container}
    Mixin base class to define a container object. The container implements the
    zope.app.container.interfaces.IContainer interface using a BTree, providing
    reasonable performance for large collections of objects.
  \end{classdesc*}

\section{\class{grok.DisplayForm}}

\section{\class{grok.EditForm}}

\section{\class{grok.Form}}

\section{\class{grok.GlobalUtility}}

  \begin{classdesc*}{grok.GlobalUtility}
    Base class to define a globally registered utility. Global utilities are
    automatically registered when a module is "grokked".

  \begin{bf}Directives:\end{bf}

  \begin{itemize}
    \item[\function{grok.implements(*interfaces)}] Required. Identifies the
    interfaces(s) the utility implements.

    \item[\function{grok.name(name)}] Optional. Identifies the name used for
    the adapter registration. If ommitted, no name will be used.

    When a name is used for the global utility registration, the global utility
    can only be retrieved by explicitely using its name.

    \item[\function{grok.provides(name)}] Maybe required. If the global utility
    implements more than one interface, \function{grok.provides} is required to
    disambiguate for what interface the global utility will be registered.
  \end{itemize}
  \end{classdesc*}

\section{\class{grok.Indexes}}

\section{grok.InstanceGrokker}

\section{\class{grok.JSON}}

\section{\class{grok.LocalUtility}}

  \begin{classdesc*}{grok.LocalUtility}
    Base class to define a utility that will be registered local to a
    \class{grok.Site} or \class{grok.Application} object by using the
    \function{grok.local_utility} directive.

  \begin{bf}Directives:\end{bf}

  \begin{itemize}
    \item[\function{grok.implements(*interfaces)}] Optional. Identifies the
    interfaces(s) the utility implements.

    \item[\function{grok.name(name)}] Optional. Identifies the name used for
    the adapter registration. If ommitted, no name will be used.

    When a name is used for the local utility registration, the local utility
    can only be retrieved by explicitely using its name.

    \item[\function{grok.provides(name)}] Maybe required. If the local utility
    implements more than one interface or if the implemented interface cannot
    be determined, \function{grok.provides} is required to disambiguate for
    what interface the local utility will be registered.
  \end{itemize}
  \end{classdesc*}

  \begin{seealso}
  Local utilities need to be registered in the context of \class{grok.Site} or
  \class{grok.Application} using the \function{grok.local_utility} directive.
  \end{seealso}

\section{\class{grok.Model}}

  Base class to define an application "content" or model object. Model objects
  provide persistence and containment.

\section{grok.ModuleGrokker}

\section{\class{grok.MultiAdapter}}

  \begin{classdesc*}{grok.MultiAdapter}
    Base class to define a multi adapter. MultiAdapters are automatically
    registered when a module is "grokked".

  \begin{bf}Directives:\end{bf}

  \begin{itemize}
    \item[\function{grok.adapts(*objects_or_interfaces)}] Required. Identifies
    the combination of types of objects or interfaces for the adaptation.

    \item[\function{grok.implements(*interfaces)}] Required. Identifies the
    interfaces(s) the adapter implements.

    \item[\function{grok.name(name)}] Optional. Identifies the name used for
    the adapter registration. If ommitted, no name will be used.

    When a name is used for the adapter registration, the adapter can only
    be retrieved by explicitely using its name.

    \item[\function{grok.provides(name)}] Maybe required. If the adapter
    implements more than one interface, \function{grok.provides} is required to
    disambiguate for what interface the adapter will be registered.
  \end{itemize}
  \end{classdesc*}

  \begin{bf}Example:\end{bf}

  \begin{verbatim}
import grok
from zope import interface

class Fireplace(grok.Model):
    pass

class Cave(grok.Model):
    pass

class IHome(interface.Interface):
    pass

class Home(grok.MultiAdapter):
    grok.adapts(Cave, Fireplace)
    grok.implements(IHome)

    def __init__(self, cave, fireplace):
        self.cave = cave
        self.fireplace = fireplace

home = IHome(cave, fireplace)
  \end{verbatim}

\section{grok.PageTemplate}

\section{grok.PageTemplateFile}

\section{grok.Permission}

  \begin{classdesc*}{grok.Permission}
    Base class to define a permission. Permissions are automatically
    registered when a module is "grokked".

  \begin{bf}Attributes:\end{bf}

  \begin{itemize}
    \item[\function{id(dotted_name)}] Required. Dotted name identifying this
      permssion. Usually prefixed by an application specific prefix.

    \item[\function{title(name)}] Required. Human readable title for this
      permission. Must be a unicode string.

    \item[\function{description(text)}] Optional. Human readable text
      describing the use of this permssion.
  \end{itemize}
  \end{classdesc*}

  \begin{bf}Example:\end{bf}

  \begin{verbatim}
import grok

class AddPainting(grok.Permission):
    id = 'cavepainter.AddPainting'
    title = u'Cave Painter Add Painting'
    description = (
        u'The permission to add a new painting in the '
        u'cave painter application')
  \end{verbatim}

  \begin{seealso}
  See alos \class{grok.Role}.
  \end{seealso}

\section{grok.Role}

  \begin{classdesc*}{grok.Role}
    Base class to define a role. Roles are automatically registered when a
    module is "grokked". A Role is a grouping of permissions.

  \begin{bf}Attributes:\end{bf}

  \begin{itemize}
    \item[\function{id(dotted_name)}] Required. Dotted name identifying this
      role. Usually prefixed by an application specific prefix.

    \item[\function{title(name)}] Required. Human readable title for this
      role. Must be a unicode string.

    \item[\function{description(text)}] Optional. Human readable text
      describing the use of this role.
  \end{itemize}

  \begin{bf}Directives:\end{bf}

  \begin{itemize}
    \item[\function{grok.permissions(*permission_ids)}] Optional. One or more
      permission ids that compise this role. Permissions need to be defined
      before using this directive.
  \end{itemize}
  \end{classdesc*}

  \begin{bf}Example:\end{bf}

  \begin{verbatim}
import grok

class AddPainting(grok.Permission):
    id = 'cavepainter.AddPainting'
    title = u'Cave Painter Add Painting'
    description = (
        u'The permission to add a new painting in the '
        u'cave painter application')

class EditPainting(grok.Permission):
    id = 'cavepainter.EditPainting'
    title = u'Cave Painter Edit Painting'
    description = (
        u'The permission to edit a existing painting in the '
        u'cave painter application')

class Painter(grok.Role):
    id = 'cavepainter.Painter'
    title = u'Cave Painter'
    description = (
        u'The Painter role in the Cave Painter application')'
    grok.permissions(
        'cavepainter.AddPainting', cavepainter.EditPainting')

  \end{verbatim}

  \begin{seealso}
  The permissions need to be registered first before defining the role that uses these permissions. See \class{grok.Permission}.
  \end{seealso}

\section{\class{grok.Site}}

  Base class to define an site object. Site objects provide persistence and
  containment.

\section{\class{grok.Traverser}}

\section{\class{grok.View}}

\section{\class{grok.XMLRPC}}


\chapter{Directives}

The \module{grok} module defines a set of directives that allow you to
configure and register your components. Most directives assume a default, based
on the environment of a module. (For example, a view will be automatically
associated with a model if the association can be made unambigously.) 

If no default can be assumed for a value, grok will explicitly tell you what is
missing and how you can provide a default or explicit assignment for the value
in question.

    \section{grok.implements}

    \section{grok.context}

    \section{grok.name}

    \section{grok.template}

    \section{grok.templatedir}

    \section{grok.templatedir}



\chapter{Decorators}

grok uses a few decorators to register functions or methods for specific
functionality.

    \section{\function{grok.subscribe} -- Register a function as a subscriber
    for an event}

    Similar to Zope 3's \function{zope.component.subscriber} decorator, the
    function where it is applied to is registered as a subscriber for an event. 




    \section{grok.action}



\chapter{Functions}

The \module{grok} module provides a number of convenience functions to aid in
common tasks.

  \section{\function{grok.getSite}}

    \begin{funcdesc}{grok.getSite}{*args}
    foobar
    \end{funcdesc}

  \section{\function{grok.notify}}

    \begin{funcdesc}{grok.notify}{*args}
    foobar
    \end{funcdesc}

  \section{\function{grok.url}}

    \begin{funcdesc}{grok.url}{request, object, \optional{, name}}
    Construct a URL for the given \var{request} and \var{object}.

    \var{name} may be a string that gets appended to the object URL. Commonly
    used to construct an URL to a particular view on the object.

    This function returns the constructed URL as a string.

    \begin{seealso}
      View classes derived from \class{grok.View} have a similar \method{url}
      method for constructing URLs.
    \end{seealso}
    \end{funcdesc}


\chapter{Events}

grok provides convenient access to a set of often-used events from Zope 3.
Those events include object and containment events. All events are available as
interface and implemented class.

    \section{grok.IObjectCreatedEvent}

    \section{grok.IObjectModifiedEvent}

    \section{grok.IObjectCopiedEvent}

    \section{grok.IObjectAddedEvent}

    \section{grok.IObjectMovedEvent}

    \section{grok.IObjectRemovedEvent}

    \section{grok.IContainerModifiedEvent}



\include{exceptions}

\end{document}
